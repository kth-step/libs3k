\documentclass[a4paper,11pt]{article}
\usepackage[top=1in, bottom=1in]{geometry}
\usepackage{hyperref}
\usepackage{listings}
\usepackage{xcolor}
\usepackage{amsmath}
\usepackage{booktabs}
\usepackage{enumitem}
\usepackage{longtable}
\usepackage{tcolorbox}
\tcbuselibrary{breakable}

\title{S3K: System Calls Documentation}
\author{Henrik Karlsson}
\date{\today}

\newcommand{\syscall}[1]{\texttt{#1}}


\newenvironment{syscalldoc}[1]{
  \begin{tcolorbox}[breakable,title=\subsection{\syscall{#1}}]
  \begin{description}[leftmargin=!,style=nextline,noitemsep]
}{
  \end{description}
  \end{tcolorbox}
}

\lstset{basicstyle=\ttfamily,breaklines=true,breakindent=2em}

\begin{document}

\maketitle
\tableofcontents

\section{Introduction to System Calls}
% Brief explanation of system calls

\section{List of System Calls}

%\begin{longtable}{ll}
%    \toprule
%    \textbf{System Call} & \textbf{Description} \\
%    \midrule
%    \syscall{s3k\_get\_pid()} & Fetches the current process's ID \\
%    \syscall{s3k\_get\_time()} & Retrieves the current system time \\
%    \syscall{s3k\_get\_timeout()} & Retrieves the current timeout value \\
%    \syscall{s3k\_reg\_read()} & Reads the value of a specified register \\
%    \syscall{s3k\_reg\_write()} & Writes a value to a specified register \\
%    \syscall{s3k\_sync()} & Synchronizes the system state \\
%    \syscall{s3k\_sleep()} & Pauses execution for a specified time \\
%    \syscall{s3k\_cap\_read()} & Reads a capability from a specified index \\
%    \syscall{s3k\_cap\_move()} & Moves a capability from one index to another \\
%    \syscall{s3k\_cap\_delete()} & Deletes a capability at a specified index \\
%    \syscall{s3k\_cap\_revoke()} & Revokes a capability at a specified index \\
%    \syscall{s3k\_cap\_derive()} & Derives a new capability from an existing one \\
%    \syscall{s3k\_pmp\_load()} & Loads a PMP configuration \\
%    \syscall{s3k\_pmp\_unload()} & Unloads a PMP configuration \\
%    \syscall{s3k\_mon\_suspend()} & Suspends a monitor \\
%    \syscall{s3k\_mon\_resume()} & Resumes a monitor \\
%    \syscall{s3k\_mon\_state\_get()} & Retrieves the state of a monitor \\
%    \syscall{s3k\_mon\_yield()} & Yields control to a monitor \\
%    \syscall{s3k\_mon\_reg\_read()} & Reads a register value for a monitor \\
%    \syscall{s3k\_mon\_reg\_write()} & Writes a register value for a monitor \\
%    \syscall{s3k\_mon\_cap\_read()} & Reads a capability for a monitor \\
%    \syscall{s3k\_mon\_cap\_send()} & Sends a capability to a monitor \\
%    \syscall{s3k\_mon\_pmp\_load()} & Loads a PMP configuration for a monitor \\
%    \syscall{s3k\_mon\_pmp\_unload()} & Unloads a PMP configuration for a monitor \\
%    \syscall{s3k\_sock\_send()} & Sends a message via a socket \\
%    \syscall{s3k\_sock\_recv()} & Receives a message via a socket \\
%    \syscall{s3k\_sock\_sendrecv()} & Sends and receives a message via a socket \\
%    \bottomrule
%\end{longtable}

\newpage
\section{System Call Description}

\begin{syscalldoc}{s3k\_get\_pid}
  \item[Syntax] \lstinline{s3k_pid_t s3k_get_pid(void)}
  \item[Description] Fetches the process ID of the caller.
  \item[Returns] The process ID of the caller.
\end{syscalldoc}

\begin{syscalldoc}{s3k\_get\_time}
  \item[Syntax] \lstinline{uint64_t s3k_get_time(void)}
  \item[Description] Fetches the current value of the real-time clock.
  \item[Returns] The current value of the real-time clock.
  \item[Notes] The frequency of the RTC is hardware dependant.
\end{syscalldoc}

\begin{syscalldoc}{s3k\_get\_timeout}
  \item[Syntax] \lstinline{uint64_t s3k_get_timeout(void)}
  \item[Description] Fetches the preemption time, which indicates how long the current process can run before being preempted.
  \item[Returns] The current value of the preemption time.
  \item[Notes] The frequency of the RTC is hardware dependant.
\end{syscalldoc}

\begin{syscalldoc}{s3k\_reg\_read}
  \item[Syntax] \lstinline{uint64_t s3k_reg_read(s3k_reg_t reg)}
  \item[Description] Reads the value of the specified register \verb|reg|. This system call is primarily used to read S3K's virtual registers but can also read standard RISC-V registers.
  \item[Parameters]
    \begin{description}
      \item[]
      \item[\texttt{reg}] The register to read. Should be one of S3K's virtual registers or a standard RISC-V register.
    \end{description}
  \item[Returns] The value of the specified register \verb|reg|. Returns 0 if \verb|reg| is invalid.
  \item[Notes] Returns 0 if the specified register is invalid. Ensure that the register being read is valid.
\end{syscalldoc}

\begin{syscalldoc}{s3k\_reg\_write}
  \item[Syntax] \lstinline{uint64_t s3k_reg_write(s3k_reg_t reg, uint64_t val)}
  \item[Description] Writes the value \verb|val| to the specified register \verb|reg|. This system call is primarily used to write to S3K's virtual registers but can also write to standard RISC-V registers.
  \item[Parameters]
    \begin{description}
      \item[]
      \item[\texttt{reg}] The register to write to. Should be one of S3K's virtual registers or a standard RISC-V register.
      \item[\texttt{val}] The value to write to the register.
    \end{description}
  \item[Returns] The value of the specified register \verb|reg| before the write operation. Returns 0 if \verb|reg| is invalid.
  \item[Notes] Returns 0 if the specified register is invalid. Ensure that the register being written to is valid.
\end{syscalldoc}

\begin{syscalldoc}{s3k\_sync}
  \item[Syntax] \lstinline{void s3k_sync(void)}
  \item[Description] Synchronizes the process's context with capabilities. This ensures that any changes to capabilities are reflected in the process's execution context.
  \item[Returns] This function does not return a value.
  \item[Notes] This function should be called after modifying capabilities such as time slices or PMP to ensure that the changes take effect immediately.
\end{syscalldoc}

\begin{syscalldoc}{s3k\_sleep}
  \item[Syntax] \lstinline{void s3k_sleep(uint64_t time)}
  \item[Description] Sets the process to sleep until the real-time clock (RTC) reaches the specified \verb|time|. If \verb|time| is 0, the process sleeps until the next timer preemption, as determined by \verb|s3k_get_timeout()|.
  \item[Parameters]
    \begin{description}
      \item[]
      \item[\texttt{time}] The time at which the process should wake up. If 0, the process sleeps until the next timer preemption.
    \end{description}
  \item[Returns] This function does not return a value.
  \item[Notes] Ensure that the \verb|time| value is valid and represents a future point in time. If \verb|time| is in the past, the process will wake up immediately.
\end{syscalldoc}

\begin{syscalldoc}{s3k\_cap\_read}
  \item[Syntax] \lstinline{s3k_err_t s3k_cap_read(s3k_cidx_t idx, s3k_cap_t *cap)}
  \item[Description] Reads the description of the capability at index \verb|idx| in the caller's capability table. This function is used to retrieve information about a specific capability.
  \item[Parameters]
    \begin{description}
      \item[]
      \item[\texttt{idx}] The index in the caller's capability table.
      \item[\texttt{cap}] A pointer to a buffer where the capability description will be stored.
    \end{description}
  \item[Returns]
    \begin{description}
      \item[]
      \item[\texttt{S3K\_SUCCESS}] If the capability is successfully read.
      \item[\texttt{S3K\_ERR\_INVALID\_INDEX}] If \verb|idx| is out of range.
      \item[\texttt{S3K\_ERR\_EMPTY}] If there is no capability at \verb|idx|.
    \end{description}
  \item[Notes] Ensure that the \verb|cap| buffer is properly allocated and can hold the capability description. This function is useful for inspecting capabilities before performing operations that depend on them.
\end{syscalldoc}

\begin{syscalldoc}{s3k\_cap\_move}
  \item[Syntax] \lstinline{s3k_err_t s3k_cap_move(s3k_cidx_t src, s3k_cidx_t dst)}

  \item[Description] Moves a capability from index \verb|src| to \verb|dst| in the caller's capability table. This function is used to reorganize capabilities within the table.

  \item[Parameters]
    \begin{description}
      \item[]
      \item[\texttt{src}] The index in the caller's capability table from which the capability will be moved.
      \item[\texttt{dst}] The index in the caller's capability table to which the capability will be moved.
    \end{description}

  \item[Returns]
    \begin{description}
      \item[]
      \item[\texttt{S3K\_SUCCESS}] If the capability is successfully moved.
      \item[\texttt{S3K\_ERR\_INVALID\_INDEX}] If \verb|src| or \verb|dst| is out of range.
      \item[\texttt{S3K\_ERR\_SRC\_EMPTY}] If there is no capability at \verb|src|.
      \item[\texttt{S3K\_ERR\_DST\_OCCUPIED}] If there is already a capability at \verb|dst|.
    \end{description}

  \item[Notes] Ensure that the destination index \verb|dst| is not occupied before moving the capability. This function is useful for reorganizing capabilities within the table.
\end{syscalldoc}

\begin{syscalldoc}{s3k\_cap\_delete}
  \item[Syntax] \lstinline{s3k_err_t s3k_cap_delete(s3k_cidx_t idx)}

  \item[Description] Deletes a capability at index \verb|idx| in the caller's capability table. This function is used to remove a capability from the table, freeing up the index for future use.

  \item[Parameters]
    \begin{description}
      \item[]
      \item[\texttt{idx}] The index in the caller's capability table from which the capability will be deleted.
    \end{description}

  \item[Returns]
    \begin{description}
      \item[]
      \item[\texttt{S3K\_SUCCESS}] If the capability is successfully deleted.
      \item[\texttt{S3K\_ERR\_INVALID\_INDEX}] If \verb|idx| is out of range.
      \item[\texttt{S3K\_ERR\_EMPTY}] If there is no capability at \verb|idx|.
    \end{description}

  \item[Notes] Ensure that the index \verb|idx| is valid and that there is a capability present at that index before attempting to delete it. This function is useful for managing the capability table by removing unused or unwanted capabilities.
\end{syscalldoc}

\begin{syscalldoc}{s3k\_cap\_revoke}
  \item[Syntax] \lstinline{s3k_err_t s3k_cap_revoke(s3k_cidx_t idx)}

  \item[Description] Revokes the children of the capability at index \verb|idx| in the caller's capability table. This function is used to reclaim resources that have been granted to child capabilities.

  \item[Parameters]
    \begin{description}
      \item[]
      \item[\texttt{idx}] The index in the caller's capability table from which the capability's children will be revoked.
    \end{description}

  \item[Returns]
    \begin{description}
      \item[]
      \item[\texttt{S3K\_SUCCESS}] If the children are successfully revoked.
      \item[\texttt{S3K\_ERR\_INVALID\_INDEX}] If \verb|idx| is out of range.
      \item[\texttt{S3K\_ERR\_EMPTY}] If there is no capability at \verb|idx|.
    \end{description}

  \item[Notes] Ensure that the index \verb|idx| is valid and that there is a capability present at that index before attempting to revoke its children. This function is useful for managing the capability table by reclaiming resources from child capabilities.
\end{syscalldoc}

\begin{syscalldoc}{s3k\_cap\_derive}
  \item[Syntax] \lstinline{s3k_err_t s3k_cap_derive(s3k_cidx_t src, s3k_cidx_t dst, s3k_cap_t newcap)}
  \item[Description]
	  Derives a new capability \verb|newcap| from the capability at index \verb|src| in the caller's capability table. The new capability is placed at index \verb|dst| in the caller's capability table.

  \item[Parameters]
    \begin{description}
      \item[]
      \item[\texttt{src}] The index in the caller's capability table from which the new capability will be derived.
      \item[\texttt{dst}] The index in the caller's capability table where the new capability will be placed.
      \item[\texttt{newcap}] The new capability to be derived and placed at \verb|dst|.
    \end{description}

  \item[Returns]
    \begin{description}
      \item[]
      \item[\texttt{S3K\_SUCCESS}] If the new capability is successfully derived and placed.
      \item[\texttt{S3K\_ERR\_INVALID\_INDEX}] If \verb|src| or \verb|dst| is out of range.
      \item[\texttt{S3K\_ERR\_SRC\_EMPTY}] If there is no capability at \verb|src|.
      \item[\texttt{S3K\_ERR\_DST\_OCCUPIED}] If there is already a capability at \verb|dst|.
      \item[\texttt{S3K\_ERR\_INVALID\_DERIVATION}] If \verb|newcap| cannot be derived from the capability at index \verb|src|.
    \end{description}

  \item[Notes]
	  Ensure that the indices \verb|src| and \verb|dst| are valid and that there is a capability present at \verb|src| before attempting to derive a new capability. This function is useful for creating new capabilities based on existing ones.
\end{syscalldoc}

\begin{syscalldoc}{s3k\_pmp\_load}
  \item[Syntax] \lstinline{s3k_err_t s3k_pmp_load(s3k_cidx_t pmpidx, s3k_pmp_slot_t pmpslot)}

  \item[Description] Loads a PMP configuration from the capability at index \verb|pmpidx| in the caller's capability table into the specified PMP slot \verb|pmpslot|. This function is used to configure the Physical Memory Protection (PMP) settings for the caller.

  \item[Parameters]
    \begin{description}
      \item[]
      \item[\texttt{pmpidx}] The index in the caller's capability table where the PMP capability resides.
      \item[\texttt{pmpslot}] The PMP slot of the caller to which the PMP configuration is written.
    \end{description}

  \item[Returns]
    \begin{description}
      \item[]
      \item[\texttt{S3K\_SUCCESS}] If the PMP slot was successfully configured using the PMP capability.
      \item[\texttt{S3K\_ERR\_EMPTY}] If there is no capability at \verb|pmpidx|.
      \item[\texttt{S3K\_ERR\_INVALID\_INDEX}] If \verb|pmpidx| is out of range.
      \item[\texttt{S3K\_ERR\_INVALID\_PMP}] If the capability at \verb|pmpidx| is not an unused PMP capability.
      \item[\texttt{S3K\_ERR\_INVALID\_SLOT}] If \verb|pmpslot| is out of range.
      \item[\texttt{S3K\_ERR\_DST\_OCCUPIED}] If the PMP slot \verb|pmpslot| has an existing configuration.
    \end{description}

  \item[Notes] Ensure that the index \verb|pmpidx| is valid and that there is a PMP capability present at that index before attempting to load the PMP configuration. This function is useful for configuring memory protection settings.
\end{syscalldoc}


\begin{syscalldoc}{s3k\_pmp\_unload}
  \item[Syntax] \lstinline{s3k_err_t s3k_pmp_unload(s3k_cidx_t pmpidx)}

  \item[Description] 
	  Unloads a PMP configuration from the capability at index \verb|pmpidx| in the caller's capability table. This function is used to clear the Physical Memory Protection (PMP) settings for the caller.

  \item[Parameters]
    \begin{description}
      \item[]
      \item[\texttt{pmpidx}] The index in the caller's capability table where the PMP capability resides.
    \end{description}

  \item[Returns]
    \begin{description}
      \item[]
      \item[\texttt{S3K\_SUCCESS}] If a PMP slot was successfully cleared using the PMP capability.
      \item[\texttt{S3K\_ERR\_EMPTY}] If there is no capability at \verb|pmpidx|.
      \item[\texttt{S3K\_ERR\_INVALID\_INDEX}] If \verb|pmpidx| is out of range.
      \item[\texttt{S3K\_ERR\_INVALID\_PMP}] If the capability at \verb|pmpidx| is not an used PMP capability.
    \end{description}

    \item[Notes] 
	    Ensure that the index \verb|pmpidx| is valid and that there is a PMP capability present at that index before attempting to unload the PMP configuration. This function is useful for clearing memory protection settings.
\end{syscalldoc}

\end{document}
